\chapter{Introduction}

%#####################################################################
\section{Historical overview}

% As will be shown farther Quantum key distribution (QKD) allows two
% parties to communicate in absolute security based on the fundamental
% laws of physics.

% Stephen Wiesner being the student of the Colombian university, in 1970
% has submitted article under the coding theory to magazine IEEE
% Information Theory, but it has not been published, as the assumptions
% stated in it seemed fantastic, instead of scientific. In [[1]]
% possibility of use of quantum states for protection of monetary
% banknotes has been described. Wiesner has suggested to integrate in
% each banknote 20 so-called light traps, and to place in each of them
% one photon polarized in strictly certain condition. Each banknote was
% marked by special serial number which concluded the information on
% position of the polarizing photon filter. As a result of it if
% different from sated filter was used the combination of the polarized
% photons was erased. But at that point in time technological
% development did not allow even to argue on such possibilities. However
% in 1983 his work ? Conjugate coding ? has been published in SIGACT
% News and was highly appreciated in scientific circles.

% In a consequence on the basis of principles Wiesner?s work, scientists
% Charles Bennett from IBM and Gilles Brassard from the Montreal
% University have developed a principles of message coding and
% transfer. They had been made a presentation ?Quantum cryptography:
% public key distribution and coin tossing ? at conference IEEE
% International Conference on Computers, Systems, and Signal
% Processing. Protocol described in [[2]] is recognized subsequently by
% the first and base protocol of quantum cryptography and has been named
% in honour of its founders BB84.

% At this time Artur Ekert worked over the protocol of the quantum
% cryptography, based on the quantum states entanglement
% [[3]]. Publication of results of this work has taken place in
% 1991. This protocol based on principles of Einstein-Podolsky-Rosenberg
% paradox.

% Throughout twenty five years, the quantum cryptography has passed a
% way from theoretical researches and the proof of the basic theories to
% the commercial systems using an optical fibre for transfer on distance
% of tens of kilometers.

% In the first experimental demonstration of quantum key distribution in
% 1989 [[4]], transfer was carried out through open space on distance of
% thirty centimeters. Further these experiments have been done with use
% of an optical fiber as media of distribution. After first
% experiments. in Geneva, with use of an optical fiber with length of
% 1,1 km [[5]], in 1995 distance of transfer have been increased to 23
% km with optical fiber laid under water [[6]]. Approximately at the
% same time, Townsend from British Telecom had been showed transfer on
% 30 km [[7]]. Later he has increased range to 50 km [[8]]. In 2001,
% Hisket in the United Kingdom had been made a demonstration of transfer
% on distance of 80 km [[9]]. RECORD[MAU2]

% Firs commercial QKD system was presented at CeBIT-2002 by GAP-Optique
% company. Scientists managed to create compact enough and reliable
% device. The system settled down in two 19-inch racks and could work
% without any adjustment right after connections to the personal
% computer. Bidirectional land and air fiber-optical connection has been
% established with its help between the cities of Geneva and Luzanna the
% distance between which is 67 km [[10]].

% The next years such commercial monsters as Toshiba, NEC, IBM, Hewlett
% Packard, Mitsubishi were connected to designing and manufacturing of
% quantum cryptography systems. But along with them the small but
% hi-tech companies began to appear in the market: MagiQ, Id Quantique,
% Smart Quantum. All companies set forth above make systems with phase
% coding of qubits.


% OPEN SPACE?[MAU3]

% Thus, less than for 50 years the quantum cryptography has passed a way
% from idea to an embodiment in commercial system of quantum key
% distribution. The basic consumers of systems of quantum cryptography
% is the Ministries of Defense, the Ministries for Foreign Affairs and
% large commercial companys. Currently high cost of quantum systems of
% distribution of keys limits their mass application.


%#####################################################################
\section{Quantunm cryptography basics}


% There is two main directions in quantum cryptography.

% The first direction is based on coding of a quantum state of a single
% particle and is based on an principle of impossibility to reliably
% distinguish two non-orthogonal quantum states.

% Any state of two-level quantum system can be represented as linear
% combination(superposition) of its states:

           
% The numbers and are complex numbers and

           
% Laws of quantum mechanics tell us that two quantum states


% And


% cant be reliably distinguish If is not true, i.e. this states is
% orthogonal.

% Security of the first direction is based on the no-cloning theorem,
% and it is one of the chief differences between quantum and classical
% information .Thanks to unitarity and linearity of quantum mechanics,
% it is impossible to create an exact copy of an unknown quantum state
% without influence on an initial condition. Let, for example, sender
% (Alice) and the receiver (Bob) use the two-level quantum systems for
% an information transfer, by coding conditions of these systems. If the
% eavsdropper (Eve) intercepts a data carrier, sended by Alice, measures
% its condition and sends further to the Bob, that condition of this
% carrier will be other, than before measurement. Thus, interception of
% the quantum channel leads to errors of transfer which can be found out
% by legal users.

% The base protocol of one-particle quantum cryptography is BB84
% protocol [].

% The second direction is based on entanglement of quantum states. Two
% quantum-mechanical systems (including space spatially) can be in a
% correlation, so measuring of chosen characteristic of one system will
% define result of measuring of same characteristic of another. Not a
% one of the entangled systems is in determined condition. That?s why
% entanglement state can?t be defined as direct multiplication of
% systems states. State of two particles with ? spin is the example of
% entangled state

% .

           
% Measurement performed on one subsystem gives with equal probability
% state or . The state of other subsystem will be opposite, i.e.  if the
% result of measurement on the first system was vice- versa.

% The base protocol for systems entanglement quantum cryptography is
% EPR(Einstein-Podolsky-Rosen) protocol. Another name of this protocol
% is E91[].

% Base principles of these two directions have laid down in a basis of
% working out of all protocols of quantum key distribution.



%#####################################################################
\section{BB84 Basics}

% There is a set of protocols of quantum cryptography based on an
% information transfer by single photon state coding, for example: BB84,
% B92 [28], ??84 (4+2) [29], with six conditions [30], Goldenberga -
% Vajdmana [31], Koashi-Imoto [32] and their modifications. We will look
% closer at BB84 protocol because it is basic protocol for our QKD
% setup.

% In BB84 protocol 4 quantum states of photons, for example, a direction
% of a vector of polarization are used. Depends on chosen bit Alice use
% 90? or 135? for ?1?, 45? or 0? for ?0?. One pair of quantum conditions
% corresponds to and and belongs to basis ?+?. Other pair of quantum
% conditions corresponds to and and belongs to basis ? ?. In both basis
% states are orthogonal, but states from different basis are in pairs
% not orthogonal (nonorthogonality is necessary for detecting of
% attempts eavsdropping).

% Quantum states of system can be write down in next way

           
% Here and corresponds to ?0? and ?1? in ?+? basis, and and corresponds
% to same values in ??? basis.

% Bases are turned from each other on 45 ? (Figure 1).


% Figure1



% Stages of key distribution:



% 1) Alice in a random way chooses one of bases. Then in basis chooses
% randomly one of the states, corresponding 0 or 1 and sends photons
% (figure 2)


% Figure 2 - Photons with various polarization

% 2) Bob randomly and irrespective of Alice chooses for each arriving
% photon: rectilinear (+) or diagonal ( ) basis (Figure 3):


% ??????? 3 - The chosen type of measurements

% Then Bob saves results of measurements:

% Drawing 4 - Results of measurements


% 3) Bob on an public communication channel informs, what type of
% measurements has been used for each photon that is what has been
% chosen basis, but results of measurements remain in a secret;

% 4) Alice informs Bob on the public communication channel, what
% measurements have been chosen according to initial basis of Alice
% (Figure 5):


% Figure 5 - Cases of correct gaugings

% 5) Further users leave only those cases in which the chosen bases have
% coincided. These cases translate in bits (0 and 1), and thus receive a
% key(Figure 6):


% Figure 6 - Reception of key sequence by results of correct gaugings

% The number of cases in which the chosen bases have coincided, will be
% average half of length of initial sequence, i.e.  (example in Table
% 1).



% Table 1



% Thus, as a result of transmission of a key in case of absence of noise
% and distortions
% 50 of photons will correctly registered by Bob on the average.

% However ideal quantum channels does not exist and for formation of a
% confidential key it is necessary to carry out additional procedures of
% error correction and privacy amplification.

%#####################################################################
\section{BB84 With decoy states}

%#####################################################################
\section{QKD Experimental realization}
